\documentclass[12pt]{article}
%\documentclass{exam}
\usepackage{tikz, multicol, graphicx, etoolbox, enumerate, setspace, relsize, mathrsfs, verbatim}
\usepackage{amsmath, amsfonts, amssymb, amsthm, epsfig, epstopdf, titling, url}
\usepackage[margin=1.00in]{geometry}
\usepackage{xspace}
\newcommand{\matlab}{\textsc{Matlab}\xspace}
%======================================================================
% End of preamble
%======================================================================
\begin{document}
\thispagestyle{empty}
\raggedright

\hfill Reflection 2  % the reflection number

\hfill Stephen Beckstrand  % your name

\hfill A02311346 % your A#

Reflection on \emph{The Empty Set}.  % Leave this in your final document.

\begin{flushleft}
Before explaining the empty set, I think it would be helpful to first explain the basics of Set Theory, as it will help when trying to understand what the empty set is and what purpose it serves in mathematics.
\end{flushleft}

\begin{flushleft}
First, we will start with what a \textit{set} is. A set is basically a group of elements and is denoted as curly brackets with each element being separated by a comma. The elements can be specific objects, or they can be a bit arbitrary or broad. For example, you could have a set of the numbers 1, 2, 3, denoted as \{1, 2, 3\}, or you could have a set of all real numbers, or a set of all colors, or even a set of all humans named Stephen. Generally speaking, sets are used to help define and classify elements.
\end{flushleft}

\begin{flushleft}
Next, I should explain the concept of a \textit{subset} . A set is considered to be a subset of another set if every element in your set exists in the set it is being compared to. For example, \{1 2\} would be considered a subset of \{1, 2, 3, 4\} since the set we are comparing against, contains every element that is in our original set.
\end{flushleft}

\begin{flushleft}
Finally, to get to the topic of interest, \textit{the empty set} is a set that does not contain any elements. It is denoted as \{ \} or ∅ (A zero with a slash through it). The idea of an empty set sounds simple enough but it is important that it is defined. Let me try to explain and perhaps give some examples of why it is important. One of the primary benefits of using sets is the ability to combine or compare multiple sets with each other. For example, we might try to find all elements that two sets share with each other, or determine what items do not exist in one set, when compared to another. In some cases, the answer is a set without any elements, and we need a way to define this. This also supports the idea that the empty set is a subset of every other set. For example, lets say you have two sets, one assigned to the variable A and one assigned to the variable B. Let A = \{1, 2, 3, 4\} and B = \{5, 6, 7\} . If we were trying to find the set that contains all elements shared between these two sets, we would get the empty set. Another example would be if we had two sets that were equal, but this wasn't immediately obvious to us, so we tried to find what elements do not exist in one, but exist in the other, we would again get the empty set. To flesh this example out further, Say we have the two following sets: A = \{ 2, 3, 5, 7, 11, 13, 17, 19 \}, B = \{ prime numbers that are equal to less than 20 \}
\end{flushleft}

\end{document}
