\documentclass[12pt]{article}
%\documentclass{exam}
\usepackage{tikz, multicol, graphicx, etoolbox, enumerate, setspace, relsize, mathrsfs, verbatim}
\usepackage{amsmath, amsfonts, amssymb, amsthm, epsfig, epstopdf, titling, url}
\usepackage[margin=1.00in]{geometry}
\usepackage{xspace}
\newcommand{\matlab}{\textsc{Matlab}\xspace}
%======================================================================
% End of preamble
%======================================================================
\begin{document}
\thispagestyle{empty}
\raggedright

\hfill \today    % the date you compiled the document.

\hfill Reflection 1  % the reflection number

\hfill Stephen Beckstrand  % your name

\hfill A02311346 % your A#

Reflection on \emph{Helping Undergraduates Learn to Read Mathematics}.  % Leave this in your final document.

\vspace{1em}  % leave this as a space between the heading and your reflection.

\begin{flushleft}
While reading Ashley Reiter's article \emph{Helping Undergraduates Learn to Read Mathematics}, my thoughts were brought to my first mathematics course taken in higher education; College Algebra (MATH-1050). The course consisted of four modules and at the end of each module, we were expected to write a several-page essay explaining the primary concepts and principles taught within that module. An important component to these essays was that they would be peer-reviewed by three other students to give feedback, whether that be for corrections or general appreciation. In fact, as part of the assignment description, we were instructed to write the essays as if we were trying to briefly explain and teach these concepts to somebody else, as opposed to just reiterating what we learned back to the professor. At first, I was not receptive to the idea of writing mathematical essays as this approach to learning differed quite a bit from how I was accustomed to my mathematics classes being taught. To be honest, up to this point I had not considered mathematics to be a subject that lent itself to have essays written about it. At least not when considering teaching individual concepts. However, as I was reading other student's essays and writing my own I came to appreciate that a well-written paper can be just as helpful in learning mathematics as a standard approach of a lecture with examples.
\end{flushleft}

\begin{flushleft}
Moving forward, as I attended lectures for other math courses, I found that having exposure to both writing and reading notes on mathematical concepts helped me to study material better. To clarify, up to this point I was used to referring to a textbook entirely for material and all my written work consisted of working out the problems. Learning how to effectively take notes not only helped retain the information more effectively but also allowed me to easily come back to the concept and read it again if I needed a refresher. That alone, I think, shows how effective writing and reading in math can be helpful.
\end{flushleft}

\begin{flushleft}
Lastly, a thought I had while reading this paper and its suggested importance of effectively reading mathematics; relying solely on how math is traditionally taught, such as through lectures, examples, and then assigned problems is likely to create a dependency on somebody else being readily available to teach you. Alternatively, being able to effectively read mathematics creates opportunities for learning from decades and even centuries of teachings and thought from others.
\end{flushleft}


\end{document}
